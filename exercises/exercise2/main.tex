\documentclass{article}
\usepackage[utf8]{inputenc}
\usepackage{amsmath}
\usepackage{amssymb}
\usepackage{physics}
\usepackage{gensymb}
\usepackage{esvect}


\title{Solutions to the Quantum Circuits and Devices by Prof. Ielmini exercise set \#2}
\author{}
\date{January 2025}

\begin{document}

\maketitle

\section{Exercise 1}
\subsection{Question}
Derive the Pauli operator for direction $\vv{n}$ described by $\theta = \frac{\pi}{2}$, $\phi = \frac{\pi}{4}$, calculate its corresponding eigenvectors and eigenvalues, and plot them on the Bloch sphere.
\subsection{Answer}
$$\vv{n} = \sin{\theta}\cos{\phi}\hat{x} + \sin{\theta}\sin{\phi}\hat{y} + \cos{\theta}\hat{z} \implies \sigma_{\hat{n}} = \vv{n}.\vv{\sigma} = $$

$$n_x\sigma_x + n_y\sigma_y + n_z\sigma_z = \frac{\sqrt{2}}{2}\sigma_x + \frac{\sqrt{2}}{2}\sigma_y = $$

$$\begin{bmatrix}
    0 & \frac{\sqrt{2}}{2} - i\frac{\sqrt{2}}{2} \\
    \frac{\sqrt{2}}{2} + i\frac{\sqrt{2}}{2} & 0
\end{bmatrix}$$
The expectation is that:
$$\ket{n_+} = \cos{\frac{\theta}{2}}\ket{0} + e^{i\phi}\sin{\frac{\theta}{2}}\ket{1} = \frac{1}{\sqrt{2}}\ket{0} + \frac{1}{\sqrt{2}}e^{i\frac{\pi}{4}}\ket{1}$$
$$\ket{n_-} = \sin{\frac{\theta}{2}}\ket{0} - e^{i\phi}\cos{\frac{\theta}{2}}\ket{1} = \frac{1}{\sqrt{2}}\ket{0} - e^{i\frac{\pi}{4}}\frac{1}{\sqrt{2}}\ket{1}$$
Let's investigate if we'll meet our expectations. The eigenvalues and vectors can be found:
$$\sigma_{\hat{n}}\ket{n} = \lambda \ket{n} \equiv \det{\begin{bmatrix}
    -\lambda & \frac{\sqrt{2}}{2} - i\frac{\sqrt{2}}{2} \\
    \frac{\sqrt{2}}{2} + i\frac{\sqrt{2}}{2} & -\lambda
\end{bmatrix}} = 0 \implies$$
$$\lambda^2 - (\frac{\sqrt{2}}{2} - i\frac{\sqrt{2}}{2})(\frac{\sqrt{2}}{2} + i\frac{\sqrt{2}}{2}) = \lambda^2 - 1 = 0 \implies$$
$$\lambda = \pm 1$$
We call the eigenstate associated to the $\pm 1$ eigenvalue with $\ket{n_{\pm}}$. Therefore:
$$\sigma_{\hat{n}} \ket{n_\pm} = \pm \ket{n_\pm} \equiv$$
$$\begin{bmatrix}
    0 & \frac{\sqrt{2}}{2} - i\frac{\sqrt{2}}{2} \\
    \frac{\sqrt{2}}{2} + i\frac{\sqrt{2}}{2} & 0
\end{bmatrix}\begin{bmatrix}
    n_1 \\
    n_2
\end{bmatrix} = \pm \begin{bmatrix}
    n_1 \\
    n_2
\end{bmatrix} \implies$$
$$(\frac{1}{\sqrt{2}} + i\frac{1}{\sqrt{2}})n_1 = \pm n_2 \implies e^{i\frac{\pi}{4}}n_1 = \pm n_2 \implies $$
$$\ket{n_\pm} \equiv \begin{bmatrix}
    n_1 \\ 
    \pm e^{i\frac{\pi}{4}}n_1
\end{bmatrix} = n_1\begin{bmatrix}
    1 \\ 
    \pm e^{i\frac{\pi}{4}}
\end{bmatrix}$$
As $\braket{n_\pm}{n_\pm} = 1$, then $n_1 = \frac{1}{\sqrt{2}}$, hence:
$$\ket{n_\pm} = \frac{1}{\sqrt{2}}\ket{0} \pm e^{i\frac{\pi}{4}}\frac{1}{\sqrt{2}}$$
Which are equivalent to what we expected.
\section{Exercise 2}
\subsection{Question}
Let operator $\hat{O}$ be s.t. $\hat{O}\ket{0} = \ket{-}$ and $\hat{O}\ket{1} = \ket{+}$. Find an expression for $\hat{O}$.
\subsection{Answer}
One obvious solution that comes to mind is:
$$\hat{O} = \ket{-}\bra{0} + \ket{+}\bra{1}$$
One can try to write this in $\{ \ket{0}, \ket{1}\}$ basis:
$$O_{11} = \bra{0}\hat{O}\ket{0} = \braket{0}{-} = \frac{1}{\sqrt{2}} , \quad O_{12} = \bra{0}\hat{O}\ket{1} = \braket{0}{+} = \frac{1}{\sqrt{2}}$$
$$O_{21} = \bra{1}\hat{O}\ket{0} = \braket{1}{-} = -\frac{1}{\sqrt{2}} , \quad O_{22} = \bra{1}\hat{O}\ket{1} = \braket{1}{+} = \frac{1}{\sqrt{2}}$$
Now, we test it in this basis:
$$\hat{O}\ket{0} \equiv \begin{bmatrix}
    \frac{1}{\sqrt{2}} & \frac{1}{\sqrt{2}} \\
    -\frac{1}{\sqrt{2}} & \frac{1}{\sqrt{2}}
\end{bmatrix}\begin{bmatrix}
    1 \\ 0
\end{bmatrix} = \begin{bmatrix}
    \frac{1}{\sqrt{2}} \\ -\frac{1}{\sqrt{2}}
\end{bmatrix} \equiv \ket{
-
}$$
And:
$$\hat{O}\ket{1} \equiv \begin{bmatrix}
    \frac{1}{\sqrt{2}} & \frac{1}{\sqrt{2}} \\
    -\frac{1}{\sqrt{2}} & \frac{1}{\sqrt{2}}
\end{bmatrix}\begin{bmatrix}
    0 \\ 1
\end{bmatrix} = \begin{bmatrix}
    \frac{1}{\sqrt{2}} \\ \frac{1}{\sqrt{2}}
\end{bmatrix} \equiv \ket{
+
}, \quad QED.$$


\section{Exercise 3}
\subsection{Question}
 Consider a state $\ket{\psi}$. Knowing that $P_0(\ket{\psi}) = 0.2$, $P_1(\ket{\psi}) = 0.8$, $P_0(H\ket{\psi}) = 0.6$, $P_1(H\ket{\psi}) = 0.4$, $P_0(HS^\dagger \ket{\psi}) = 0.7$, and $P_1(HS^\dagger \ket{\psi}) = 0.3$. Estimate the angles $\theta$, $\phi$ on the Bloch sphere of the state. 
\subsection{Answer}
Any generic state in the $\{ \ket{0}, \ket{1}\}$ basis can be written as:
$$\ket{\psi} = \cos{\frac{\theta}{2}}\ket{0} + e^{i\phi}\sin{\frac{\theta}{2}}\ket{1}, \quad P_0(\ket{\psi}) = |\braket{0}{\psi}|^2 = 0.2 = \cos^2{\frac{\theta}{2}}$$
$$\implies \cos{\theta} = 2\cos^2{\frac{\theta}{2}} - 1 = -0.6 \implies \theta = \pi - \arccos{0.6}$$
We define $\ket{\psi'}$:
$$H\ket{\psi} = \ket{\psi'} = \cos{\frac{\theta}{2}}H\ket{0} + e^{i\phi}\sin{\frac{\theta}{2}}H\ket{1} = \cos{\frac{\theta}{2}}\frac{\ket{0} + \ket{1}}{\sqrt{2}} + e^{i\phi}\sin{\frac{\theta}{2}}\frac{\ket{0} - \ket{1}}{\sqrt{2}}$$
$$= \frac{1}{\sqrt{2}}(\cos{\frac{\theta}{2}} + e^{i\phi}\sin{\frac{\theta}{2}})\ket{0} + \frac{1}{\sqrt{2}}(\cos{\frac{\theta}{2}} - e^{i\phi}\sin{\frac{\theta}{2}})\ket{1} \implies $$
$$P_0(\ket{\psi'}) = \frac{1}{2}(\cos{\frac{\theta}{2}}+e^{i\phi}\sin{\frac{\theta}{2}})(\cos{\frac{\theta}{2}}+e^{-i\phi}\sin{\frac{\theta}{2}}) =  $$
$$\frac{1}{2}(\cos^2{\frac{\theta}{2}} + \sin^2{\frac{\theta}{2}} + 2\sin{\frac{\theta}{2}}\cos{\frac{\theta}{2}}\cos{\phi}) = \frac{1}{2}(1 + \sin{\theta}\cos{\phi})$$
$$\implies 1 + \sqrt{1 - \cos^2{\theta}}\cos{\phi} = 1 + 0.8\cos{\phi} = 1.2 \implies \cos{\phi} = 0.25$$
Finally, according to the lecture notes $S = R_{\frac{\pi}{2}}$, where:
$$R_\phi = \begin{bmatrix}
    1 & 0 \\
    0 & e^{i\phi}
\end{bmatrix} \implies S^\dagger  = \begin{bmatrix}
    1 & 0 \\
    0 & -i
\end{bmatrix}$$
Let $\ket{\psi''} = HS^\dagger \ket{\psi}$, and more over:
$$
    HS^\dagger\ket{0} = H\ket{0} = \frac{\ket{0} + \ket{1}}{\sqrt{2}}, \quad HS^\dagger\ket{1} = -iH\ket{1} = \frac{\ket{0} - \ket{1}}{\sqrt{2}i}
\implies $$
$$\ket{\psi''} = \cos{\frac{\theta}{2}}\frac{\ket{0} + \ket{1}}{\sqrt{2}} + e^{i\phi}\sin{\frac{\theta}{2}}\frac{\ket{0} - \ket{1}}{\sqrt{2}i}$$
Which implies:
$$P_0(\ket{\psi''}) = (\frac{\cos{\frac{\theta}{2}}-ie^{i\phi}\sin{\frac{\theta}{2}}}{\sqrt{2}})(\frac{\cos{\frac{\theta}{2}}+ie^{-i\phi}\sin{\frac{\theta}{2}}}{\sqrt{2}}) = $$
$$= \frac{1}{2}(\cos^2{\frac{\theta}{2}} + \sin^2{\frac{\theta}{2}} - i\frac{e^{i\phi} - e^{-i\phi}}{2}\sin{\theta}) \implies (1 + \sin{\phi}\sin{\theta}) = 2\times 0.7$$
$$\implies 0.4 = 0.8\times \sqrt{1-0.25^2} $$ Which is impossible and there's a self contradiction in the numbers given!
\section{Exercise 4}
\subsection{Question}
Consider a two-qubit system with state $\ket{\psi} = \frac{1}{\sqrt{2}}(\ket{01} - \ket{10})$. Determine whether the state is a product state or entangled state.
\subsection{Answer}
By contradiction, let's suppose the state is a product state. The most generic product state in the basis is:
$$\ket{\phi} = (a_1\ket{0}+b_1\ket{1})\otimes (a_2\ket{0}+b_2\ket{1}) = a_1a_2\ket{00}+a_1b_2\ket{01} +b_1a_2\ket{10} + a_2b_2\ket{11}$$
We force the inequality $\ket{\psi} = \ket{\phi}$ which implies:
$$a_1a_2=0,\quad a_2b_2=0,\quad a_1b_2 \ne 0, \quad b_1a_2 \ne 0$$
Which is impossible. Therefore the state is entangled.
\section{Exercise 5}
\subsection{Question}
Consider the two-qubit circuit in Fig. 1  where input qubits $\ket{\psi_1}$, $\ket{\psi_2}$ are prepared in the $\ket{0}$ and $\ket{1}$ state respectively. Determine the equivalent circuit operator $\hat{O}$, and the output state $\ket{\psi_0}$ of the circuit.
\subsection{Answer}
The equivalent circuit is:
$$\hat{O} = \hat{U}_{CNOT}(H\otimes I)(I\otimes X) = \hat{U}_{CNOT}(H\otimes X) \equiv $$
$$\begin{bmatrix}
    1 & 0 & 0 & 0\\
    0 & 1 & 0 & 0\\
    0 & 0 & 0 & 1\\
    0 & 0 & 1 & 0
\end{bmatrix}\frac{1}{\sqrt{2}}\begin{bmatrix}
    0 & 1 & 0 & 1  \\
    1 & 0 & 1 & 0 \\
    0 & 1 & 0 & -1  \\
    1 & 0 & -1 & 0
\end{bmatrix} = \frac{1}{\sqrt{2}}\begin{bmatrix}
    0 & 1 & 0 & 1 \\
    1 & 0 & 1 & 0 \\
    1 & 0 & -1& 0 \\
    0 & 1 & 0 & -1
\end{bmatrix}$$
As $\ket{\psi_0} = \hat{O}\ket{\psi_1}\otimes \ket{\psi_2}$:
$$\ket{\psi_0} \equiv \frac{1}{\sqrt{2}}\begin{bmatrix}
    0 & 1 & 0 & 1 \\
    1 & 0 & 1 & 0 \\
    1 & 0 & -1& 0 \\
    0 & 1 & 0 & -1
\end{bmatrix} \begin{bmatrix}
    0 \\ 1 \\ 0 \\ 0
\end{bmatrix} = \frac{1}{\sqrt{2}}\begin{bmatrix}
    1 \\ 0 \\ 0 \\ 1
\end{bmatrix} \equiv \frac{1}{\sqrt{2}}(\ket{00}+\ket{11}) = \ket{\Phi^+}$$
\end{document}
